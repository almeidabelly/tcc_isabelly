
\section{Origem dos dados}

O \textit{Kaggle} \cite{kaggle} é uma plataforma online voltada para ciência de dados e aprendizado de máquina. Ela oferece competições com dados reais, notebooks interativos, datasets públicos e um fórum para colaboração entre cientistas de dados de todo o mundo. Também serve como repositório gratuito de dados para projetos acadêmicos e profissionais.

Entre os muitos conjuntos de dados hospedados no \textit{Kaggle}, destaca-se a base derivada do \textit{BoardGameGeek} (BGG) — um dos maiores bancos de dados e comunidades online dedicadas a jogos de tabuleiro. No BGG, os usuários registram suas coleções, atribuem avaliações, compartilham resenhas e discutem estratégias, o que torna o site uma referência consolidada tanto para a indústria quanto para a academia em análises envolvendo jogos de tabuleiro.


\section{Estrutura da Base de Dados}

A base de dados \textit{``Board Games Database from BoardGameGeek''}, hospedada no próprio \textit{Kaggle} e organizada pelo usuário \textit{threnjen} reune dados públicos do site \textit{BoardGameGeek}, incluindo infomrações a cerca de 22.000 jogos, 411.000 usuários e aproximadamente 19 milhões de avaliações. Segundo o próprio autor \cite{threnjen_boardgames_bgg_2025} a base é ideal para análises exploratórias, modelagem preditiva e sistemas de recomendação.

Este conjunto de dados inclui nove arquivos potenciais para exploração e/ou modelagem que são:

\begin{enumerate}

    \item \textbf{\texttt{games.csv}}: o arquivo de informações básicas com 47 classificações sobre cada um dos 22 mil jogos de tabuleiro. A chave primária é \texttt{BGGId}, que é o ID do jogo.
    
    \item \texttt{\textbf{ratings\_distribution.csv}}: inclui a distribuição completa das classificações para cada \texttt{BGGId}
    
    \item \texttt{\textbf{themes.csv}}: tabela de temas para cada \texttt{BGGId}
    
    \item \texttt{\textbf{mechanics.csv}}: tabela de mecânicas com sinalizadores binários por \texttt{BGGId}
    
    \item \texttt{\textbf{subcategories.csv}}: tabela de subcategorias com sinalizadores binários por \texttt{BGGId} 
    
    \item \texttt{\textbf{artists\_reduce}}: fornece informações sobre os artistas para cada \texttt{BGGId}. Este arquivo é reduzido para artistas com menos de 3 obras, com um sinalizador binário indicando que o jogo incluiu um artista com 3 ou mais obras.
    
    \item \texttt{\textbf{designers\_reduced}}: fornece informações do designer para cada \texttt{BGGId}. Este arquivo é reduzido para designers com menos 3 obras, com um sinalizador binário indicando que o jogo incluiu um designer com 3 ou mais obras.
    
    \item \textbf{\texttt{publishers\_reduced}}: fornece informações da editora para cada \texttt{BGGId}. Este arquivo é reduzido para editoras com menos de 3 obras, com um sinalizador binário indicando que o jogo incluiu uma editora com 3 ou mais obras.
    
    \item \textbf{\texttt{user\_ratings.csv}}: contém todas as avaliações para todos os \texttt{BGGId} com nome de usuário. Existem mais de 411 mil usuários únicos e aproximadamente 19 milhões de avaliações.

\end{enumerate}


\section{Seleção e Integração das Tabelas Utilizadas}

Embora a base do \textit{BoardGameGeek} contenha nove arquivos distintos, nem todos são igualmente relevantes para a construção de um sistema de recomendação eficiente. Considerando os objetivos deste trabalho — desenvolver um modelo de recomendação que una viabilidade computacional e representatividade dos dados — foram selecionadas apenas três tabelas principais: \texttt{\textbf{user\_ratings}}, \texttt{\textbf{games}} e \texttt{\textbf{mechanics}}.

A escolha se justifica pelos seguintes motivos:

\begin{itemize}
    
    \item \texttt{\textbf{user\_ratings:}} Essencial para recomendações baseadas em filtragem colaborativa (FC). Como essa tabela contém as avaliações dos usuários, ela permite construir a matriz usuário–item, que é fundamental para entender quem avaliou o quê e com qual nota jpá que contém a ID do jogo (\texttt{BGGId}), o nome do usuário (\texttt{Username}) e a avaliação (\texttt{Rating}).
    
    % \begin{center}
    %     \captionof{table}{Cabeçalho da tabela USER.RATINGS}
    %     \label{tab:user_ratings}
    %     \begin{tabular}{rrl}
    %         \toprule
    %         BGGId & Rating & Username \\
    %         \midrule
    %         213788 & 8.000000 & Tonydorrf \\
    %         213788 & 8.000000 & tachyon14k \\
    %         213788 & 8.000000 & Ungotter \\
    %         213788 & 8.000000 & brainlocki3 \\
    %         213788 & 8.000000 & PPMP \\
    %         \bottomrule
    %     \end{tabular}
    % \end{center}    
    
    \item \texttt{\textbf{games:}} Fornece atributos globais dos jogos, como nome (\texttt{Name}), ano de publicação (\texttt{YearPublished}), avaliação média (\texttt{AvgRating}), categorias principais, entre outros. Esses atributos são úteis na identificação dos jogos baseadas nos \texttt{BGGId} presentes na tabela de avaliações dos usuários, além de permitir a extração de características uteis para limpeza e análise exploratória dos dados.
    
    \item \texttt{\textbf{mechanics:}} É uma tabela de características binárias que indicam se um jogo possui determinadas mecânicas (como \textit{``Deck Building''}). Essas características são extremamente úteis para recomendações por similaridade de conteúdo (BC), pois permitem identificar jogos com atributos semelhantes, facilitando a recomendação de jogos que compartilham elementos específicos apreciados pelos usuários.

\end{itemize}

A combinação dessas três fontes é suficiente para representar, de forma equilibrada, as dimensões de interação (usuário–item) e conteúdo (atributos dos jogos), preservando o potencial analítico sem comprometer a eficiência computacional do sistema.  
Outros arquivos da base (\textit{Themes}, \textit{Publishers}, \textit{Designers}, entre outros) foram descartados por apresentarem menor relevância direta para o modelo ou por aumentarem significativamente o volume de dados sem ganho proporcional de informação.


%\section{Pré-processamentos dos Dados}

%Sobre a o arquivo \textbf{GAMES}, algumas variáveis foram ajustadas para facilitar a análise:

% \begin{center}
% \begin{tabular}{ll}
%     \captionof{table}{Variáveis do arquivo GAMES}
%     \label{tab:games_vars}
%     \toprule
%     Tipo de Variável & Variáveis \\
%     \midrule
%     Informação sobre os jogos & BGGId, Name, Description, YearPublished, Family, ImagePath \\
%     Métricas contínuas de avaliação/complexidade & GameWeight, AvgRating, BayesAvgRating, StdDev, LanguageEase \\
%     Número de jogadores e idade & MinPlayers, MaxPlayers, ComAgeRec, BestPlayers, GoodPlayers, MfgAgeRec \\
%     Tempo de jogo & MfgPlayTime, ComMinPlaytime, ComMaxPlaytime \\
%     Popularidade ou Engajamento & NumOwned, NumWant, NumWish, NumWeightVotes, NumUserRatings, NumComments \\
%     Estrutura ou Versão & NumAlternates, NumExpansions, NumImplementations, IsReimplementation \\
%     Origem ou financiamento & Kickstarted \\
%     Ranks oficiais & Rank:boardgame, Rank:strategygames, Rank:abstracts, Rank:familygames, Rank:thematic, Rank:cgs, Rank:wargames, Rank:partygames, Rank:childrensgames \\
%     Categóricas & Cat:Thematic, Cat:Strategy, Cat:War, Cat:Family, Cat:CGS, Cat:Abstract, Cat:Party, Cat:Childrens \\
% \bottomrule
% \end{tabular}
% \end{center}


\begin{comment}
\subsection{Construção da Matriz R}

A partir do arquivo \textbf{USER\_RATINGS}, será construída a matriz de avaliações conforme visto em \eqref{eq:matriz_user_item}, de tal forma que \( R \in \mathbb{R}^{m \times n} \), onde:
\begin{equation}
    \label{eq:matriz_r}
R = [r_{uj}], \quad
r_{uj} =
\begin{cases}
\text{nota atribuída pelo usuário } u \text{ ao jogo } j, & \text{se houver avaliação;} \\
0, & \text{caso contrário.}
\end{cases}
\end{equation}

Nessa estrutura, \( m \) representa o número de usuários únicos e \( n \) o número total de jogos. A matriz é tipicamente esparsa, já que cada usuário avalia apenas uma pequena fração dos itens disponíveis.  

O tamanho original do arquivo \textbf{USER\_RATINGS} é de aproximadamente 380 MB (sem tratamento), o que torna custoso o carregamento completo na memória e as operações iterativas de modelagem. Assim, foi necessário aplicar estratégias de pré-processamento e amostragem para tornar o sistema de recomendação mais leve e computacionalmente eficiente.



\subsection{Estratégias de Redução e Otimização}\label{sec:reducao}

Para viabilizar as análises e reduzir o custo computacional, foram adotadas as seguintes medidas:

\begin{itemize}
    \item \textbf{Filtragem temporal:} foram mantidos apenas os jogos lançados a partir da década de 1990, de modo a concentrar a análise no mercado moderno de jogos de tabuleiro, caracterizado por maior diversidade mecânica e volume de avaliações no \textit{BoardGameGeek}.

\begin{codigo}[caption = {Filtragem de GAMES.csv}, label={codigo:filtra_games},language=python, breaklines=true]
    # Jogos lançados a partir da década de 1990
    games_modern = games[games["YearPublished"] >= 1990]
\end{codigo}
    
    
    \item \textbf{Filtragem de usuários e jogos com baixa atividade:} foram excluídos usuários com poucas avaliações e jogos com reduzido número de notas, diminuindo a dimensionalidade da matriz \( R \) sem perda significativa de informação.
    
\begin{codigo}[caption = {Filtragem de Baixa Atividade}, label={codigo:filtra_bx_atvdd},language=python, breaklines=true]
    # usuários com menos de 20 avaliações são removidos     
    min_user_ratings = 20   

    # jogos com menos de 50 avaliações são removidos
    min_game_ratings = 50     

    user_counts = ratings["Username"].value_counts()
    game_counts = ratings["BGGId"].value_counts()
\end{codigo}
    
    \item \textbf{Amostragem controlada:} seleção de subconjuntos representativos de usuários e jogos, mantendo a distribuição estatística das notas e o equilíbrio entre popularidade e diversidade.
    
\begin{codigo}[caption = {Amostragem Controlada}, label={codigo:amostragem_controlada},language=python, breaklines=true]
    # Exemplo: seleciona 10.000 usuários aleatoriamente
    sampled_users = ratings["Username"].drop_duplicates().sample(10000, random_state=42)
    ratings = ratings[ratings["Username"].isin(sampled_users)]
\end{codigo}
    
    %\item \textbf{Conversão para formato esparso:} representação matricial por meio de estruturas otimizadas (como \textit{Compressed Sparse Row – CSR} ou \textit{Coordinate Format – COO}), reduzindo significativamente o uso de memória e o tempo de execução.
\end{itemize}

Essas etapas garantem a viabilidade do treinamento e da avaliação dos modelos de recomendação, mantendo a integridade estatística dos dados e a interpretabilidade dos resultados.

\end{comment}