\documentclass[aspectratio=169]{beamer}

% ======================================================
% PACOTES ESSENCIAIS
% ======================================================
\usepackage[brazil]{babel}
\usepackage[utf8]{inputenc}
\usepackage{amsmath, amsfonts, amssymb}
\usepackage{graphicx}
\usepackage{booktabs}
\usepackage{xcolor}

% ======================================================
% AJUSTES VISUAIS GERAIS
% ======================================================
\setbeamertemplate{navigation symbols}{}
\setbeamertemplate{footline}{}

% Fontes maiores (pensando em projetor)
\setbeamerfont{frametitle}{size=\Large}
\setbeamerfont{normal text}{size=\large}

% Margens internas (importante com background)
\setbeamersize{text margin left=1cm, text margin right=1cm}

% ======================================================
% BACKGROUNDS (layout institucional)
% ======================================================
\newcommand{\bgcapa}{
  \usebackgroundtemplate{
    \includegraphics[width=\paperwidth,height=\paperheight]{layout/capa.pdf}
  }
}

\newcommand{\bgbranco}{
  \usebackgroundtemplate{
    \includegraphics[width=\paperwidth,height=\paperheight]{layout/slide-branco.pdf}
  }
}

\newcommand{\bgfundo}{
  \usebackgroundtemplate{
    \includegraphics[width=\paperwidth,height=\paperheight]{layout/slide-fundo.pdf}
  }
}

% ======================================================
% COMANDO PADRÃO PARA SLIDE DE SEÇÃO
% ======================================================
\newcommand{\sectionframe}[1]{
{
\bgfundo
\begin{frame}[plain]
\vspace{0.8cm}
\hspace{0.6cm}
\Huge\textbf{#1}
\end{frame}
}
}

% ======================================================
\begin{document}

% ======================================================
% CAPA
% ======================================================
\bgcapa
\begin{frame}[plain]
% Capa inteira no background
\end{frame}

% ======================================================
% SEÇÕES DA APRESENTAÇÃO
% (serão preenchidas depois)
% ======================================================
\section{Introdução}
\sectionframe{Introdução}

% ------------------------------------------------------
% CONTEXTO
% ------------------------------------------------------
{
\bgbranco
\begin{frame}{Contexto e Motivação}
\begin{itemize}
  \item Crescimento dos jogos de tabuleiro modernos nas últimas décadas
  \item Ampliação do público e diversidade de perfis de jogadores
  \item Contribuições educacionais, sociais e cognitivas
  \item Excesso de opções dificulta a escolha de novos jogos
\end{itemize}
\end{frame}
}

% ------------------------------------------------------
% PROBLEMA E OBJETIVO
% ------------------------------------------------------
{
\bgbranco
\begin{frame}{Problema e Objetivos}
\begin{block}{Problema}
Como auxiliar usuários na descoberta de jogos de tabuleiro alinhados
às suas preferências em um cenário de grande volume de opções?
\end{block}

\vspace{0.4cm}

\begin{block}{Objetivo do Trabalho}
Avaliar e comparar diferentes abordagens de sistemas de recomendação
aplicadas ao contexto de jogos de tabuleiro modernos, utilizando dados
reais da plataforma \textit{BoardGameGeek}.
\end{block}
\end{frame}
}

\section{Metodologia}
\sectionframe{Metodologia}

% ======================================================
% O QUE SÃO SISTEMAS DE RECOMENDAÇÃO
% ======================================================
{
\bgbranco
\begin{frame}{O que são Sistemas de Recomendação}
Sistemas de recomendação são ferramentas computacionais projetadas para
auxiliar usuários na descoberta de itens de interesse em ambientes com
grande volume de opções.

% \vspace{0.4cm}

% \begin{itemize}
%   \item Utilizados em domínios como filmes, músicas, produtos e jogos
%   \item Baseiam-se em dados históricos de interação dos usuários
%   \item Buscam reduzir a sobrecarga de escolha
% \end{itemize}

\end{frame}
}

% ======================================================
% OBJETIVOS DOS SISTEMAS DE RECOMENDAÇÃO
% ======================================================
{
\bgbranco
\begin{frame}{Objetivos dos Sistemas de Recomendação}
Os principais objetivos dos sistemas de recomendação incluem:

\begin{itemize}
  \item Sugerir itens relevantes de acordo com o perfil do usuário
  \item Melhorar a experiência e satisfação do usuário
  \item Aumentar o engajamento com a plataforma
  \item Auxiliar na descoberta de novos itens
\end{itemize}
\end{frame}
}

% ======================================================
% TIPOS E ABORDAGENS
% ======================================================
{
\bgbranco
\begin{frame}{Tipos e Abordagens de Sistemas de Recomendação}
Diferentes abordagens podem ser adotadas na construção de sistemas de
recomendação, cada uma com características específicas.

\vspace{0.4cm}

\begin{itemize}
  \item Filtragem Colaborativa
  \item Sistemas Baseados em Conteúdo
  \item Modelos Baseline
  \item Sistemas Híbridos
\end{itemize}
\end{frame}
}

% ======================================================
% FILTRAGEM COLABORATIVA
% ======================================================
{
\bgbranco
\begin{frame}{Filtragem Colaborativa}
A filtragem colaborativa explora padrões de avaliação entre usuários e itens,
assumindo que usuários com comportamentos semelhantes tendem a compartilhar
preferências.

\vspace{0.3cm}

O problema é modelado por uma matriz de avaliações
\( R \in \mathbb{R}^{|U| \times |I|} \), onde \( r_{ui} \) representa a
avaliação do usuário \( u \) para o item \( i \).

\vspace{0.3cm}

A matriz \(R\) é tipicamente esparsa.
\end{frame}
}

{
\bgbranco
\begin{frame}{Filtragem Colaborativa: Predição e Top-\(k\)}
A predição da avaliação é realizada a partir dos \(k\) vizinhos mais similares
(Top-\(k\)):

\[
\hat{r}_{ui} =
\bar{r}_u +
\frac{
\sum_{v \in N_k(u)} \text{sim}(u,v)\,(r_{vi} - \bar{r}_v)
}{
\sum_{v \in N_k(u)} |\text{sim}(u,v)|
}
\]

\vspace{0.3cm}

\textbf{Vantagens}
\begin{itemize}
  \item Boa capacidade de capturar padrões coletivos
\end{itemize}

\textbf{Desvantagens}
\begin{itemize}
  \item Sensível à esparsidade e ao problema de \textit{cold-start}
\end{itemize}
\end{frame}
}

% ======================================================
% SISTEMAS BASEADOS EM CONTEÚDO
% ======================================================
{
\bgbranco
\begin{frame}{Sistemas Baseados em Conteúdo}
Os sistemas baseados em conteúdo recomendam itens semelhantes àqueles
previamente avaliados pelo usuário, utilizando características descritivas
dos itens.

\vspace{0.3cm}

A predição é obtida considerando os \(k\) itens mais similares:

\[
\hat{r}_{ui} =
\frac{
\sum_{j \in N_k(i)} \text{sim}(i,j)\, r_{uj}
}{
\sum_{j \in N_k(i)} |\text{sim}(i,j)|
}
\]

\vspace{0.3cm}

\textbf{Vantagens}
\begin{itemize}
  \item Menor impacto do \textit{cold-start} de itens
\end{itemize}

\textbf{Desvantagens}
\begin{itemize}
  \item Dependência da qualidade dos atributos dos itens
\end{itemize}
\end{frame}
}

% ======================================================
% MODELO BASELINE
% ======================================================
{
\bgbranco
\begin{frame}{Modelo Baseline}
O modelo baseline fornece uma estimativa simples das avaliações,
considerando efeitos globais e individuais:

\[
\hat{r}_{ui} = \mu + b_u + b_i
\]

\vspace{0.3cm}

\textbf{Objetivo}
\begin{itemize}
  \item Servir como referência mínima de desempenho
\end{itemize}

\textbf{Vantagens}
\begin{itemize}
  \item Simples e computacionalmente eficiente
\end{itemize}

\textbf{Desvantagens}
\begin{itemize}
  \item Não captura interações complexas entre usuários e itens
\end{itemize}
\end{frame}
}

% ======================================================
% MODELO HÍBRIDO
% ======================================================
{
\bgbranco
\begin{frame}{Modelo Híbrido}
Os sistemas híbridos combinam diferentes abordagens de recomendação,
buscando explorar suas vantagens complementares.

\[
\hat{r}_{u,i}^{(H)} =
\beta\,\hat{r}_{u,i}^{(SR_1)} +
(1-\beta)\,\hat{r}_{u,i}^{(SR_2)}, 
\text{     onde $\beta \in [0,1]$}
\]
\vspace{0.3cm}

\begin{itemize}
  \item Integra duas abordagens distintas
  \item Reduz limitações individuais dos modelos
\end{itemize}
\end{frame}
}

% ======================================================
% MEDIDAS DE SIMILARIDADE
% ======================================================
{
\bgbranco
\begin{frame}{Medidas de Similaridade}
As medidas de similaridade variam conforme o tipo de sistema de recomendação.

\vspace{0.3cm}

\begin{itemize}
  \item \textbf{Correlação de Pearson}: adequada para filtragem colaborativa
  \item \textbf{Similaridade do Cosseno}: utilizada em vetores de atributos
  \item \textbf{Distância Euclidiana}: sensível à magnitude das avaliações
  \item \textbf{Coeficiente de Jaccard}: apropriado para dados binários
\end{itemize}
\end{frame}
}

% ======================================================
% MÉTRICAS DE ACURÁCIA
% ======================================================
{
\bgbranco
\begin{frame}{Métricas de Avaliação — Acurácia}

Avaliam o erro entre os valores reais observados e as previsões geradas pelos modelos.

\vspace{0.3cm}

\textbf{Erro Médio Absoluto (MAE)}
\[
\text{MAE} =
\frac{1}{N}\sum |r_{ui} - \hat{r}_{ui}|,
\]

{\footnotesize
penaliza todos os desvios de forma linear mantendo a unidade original da variável analisada.
}

\vspace{0.3cm}

\textbf{Raiz do Erro Quadrático Médio (RMSE)}
\[
\text{RMSE} =
\sqrt{\frac{1}{N}\sum (r_{ui} - \hat{r}_{ui})^2},
\]

{\footnotesize
Atribui maior peso a erros de grande magnitude, sendo sensível a previsões muito distantes dos valores reais.
}

\end{frame}
}


% ======================================================
% MÉTRICAS DE RANQUEAMENTO
% ======================================================
{
\bgbranco
\begin{frame}{Métricas de Avaliação — Ranqueamento}

Avaliam a capacidade do modelo em priorizar corretamente os itens mais relevantes entre as primeiras posições da recomendação.

\vspace{0.4cm}

\textbf{Precision@k}
\[
\text{Precision@k} =
\frac{| \text{itens relevantes} \cap \text{top-}k |}{k},\text{}
\]

{\footnotesize
Indica a proporção de itens relevantes entre os $k$ primeiros recomendados, refletindo a qualidade do topo da lista.
}

\vspace{0.35cm}

\textbf{Recall@k}
\[
\text{Recall@k} =
\frac{| \text{itens relevantes} \cap \text{top-}k |}{|\text{itens relevantes}|}
\]

{\footnotesize
Mede a capacidade do modelo em recuperar os itens relevantes disponíveis, considerando o conjunto total de itens de interesse do usuário.
}

\end{frame}
}


\section{Dados}
\sectionframe{Dados}

% ------------------------------------------------------
% FONTE E ESTRUTURA DOS DADOS
% ------------------------------------------------------
{
\bgbranco
\begin{frame}{Base de Dados}
\begin{itemize}
  \item Fonte: plataforma \textit{BoardGameGeek}
  \item Base amplamente utilizada em estudos sobre jogos de tabuleiro
  \item Dados compostos por:
    \begin{itemize}
      \item Avaliações de usuários para jogos
      \item Informações descritivas dos jogos (categorias e mecânicas)
    \end{itemize}
  \item Estrutura típica de recomendação: matriz usuário--jogo
\end{itemize}
\end{frame}
}

\section{Resultados}
\sectionframe{Resultados}

% ======================================================
% 4.1 IMPLEMENTAÇÃO E ANÁLISE PRÁTICA
% ======================================================
{
\bgbranco
\begin{frame}{Implementação e Análise Prática}
Este capítulo apresenta os resultados obtidos a partir da implementação
dos modelos de recomendação descritos na metodologia.

\vspace{0.3cm}

\begin{itemize}
  \item Análise exploratória e pré-processamento dos dados
  \item Definição da matriz de avaliações
  \item Treinamento, parametrização e avaliação dos modelos
\end{itemize}
\end{frame}
}

% ======================================================
% 4.2 ANÁLISE EXPLORATÓRIA E PRÉ-PROCESSAMENTO
% ======================================================
{
\bgbranco
\begin{frame}{Distribuição das Notas}
A distribuição das avaliações permite compreender o comportamento
geral das notas atribuídas pelos usuários.

%\vspace{0.3cm}

\centering
\includegraphics[width=0.65\textwidth]{../images/hist_notas.png}
\end{frame}
}

{
\bgbranco
\begin{frame}{Distribuição de Avaliações por Usuário}
A maioria dos usuários avalia apenas um número reduzido de jogos,
o que contribui para a esparsidade da base.

%\vspace{0.3cm}

\centering
\includegraphics[width=0.65\textwidth]{../images/dist_avaliacoes_usuario.png}
\end{frame}
}

{
\bgbranco
\begin{frame}{Distribuição de Avaliações por Jogo}
Observa-se concentração de avaliações em um subconjunto reduzido
de jogos, indicando assimetria de popularidade.

%\vspace{0.3cm}

\centering
\includegraphics[width=0.65\textwidth]{../images/dist_avaliacoes_jogo.png}
\end{frame}
}

% ======================================================
% 4.3 DEFINIÇÃO DA MATRIZ R
% ======================================================
{
\bgbranco
\begin{frame}{Matriz de Avaliações \(R\)}
Após o pré-processamento, os dados são organizados em uma matriz
usuário--jogo \(R\), utilizada como base para os modelos de recomendação.

\vspace{0.3cm}

\begin{itemize}
  \item Linhas representam usuários
  \item Colunas representam jogos
  \item Entradas correspondem às avaliações observadas
\end{itemize}
\end{frame}
}

% ======================================================
% 4.4 DIVISÃO DOS DADOS
% ======================================================
{
\bgbranco
\begin{frame}{Divisão dos Dados em Treino e Teste}
Os experimentos foram conduzidos por meio de uma divisão dos dados
em conjuntos de treinamento e teste.

\vspace{0.3cm}

\begin{itemize}
  \item Avaliação realizada em cenário offline
  \item Garantia de comparabilidade entre os modelos
\end{itemize}
\end{frame}
}

% ======================================================
% 4.5 TREINAMENTO DOS MODELOS
% ======================================================
% 4.5.1 FILTRAGEM COLABORATIVA — ACURÁCIA
{
\bgbranco
\begin{frame}{Filtragem Colaborativa — Sensibilidade ao Top-\(k\) (Acurácia)}
\vspace{0.2cm}

\begin{columns}
  \column{0.48\textwidth}
  \centering
  \includegraphics[width=\textwidth]{../images/fc_k_rmse.png}
  \small RMSE

  \column{0.48\textwidth}
  \centering
  \includegraphics[width=\textwidth]{../images/fc_k_mae.png}
  \small MAE
\end{columns}

\end{frame}
}

% 4.5.1 FILTRAGEM COLABORATIVA — RANKING
{
\bgbranco
\begin{frame}{Filtragem Colaborativa — Sensibilidade ao Top-\(k\) (Ranking)}
\vspace{0.2cm}

\begin{columns}
  \column{0.48\textwidth}
  \centering
  \includegraphics[width=\textwidth]{../images/fc_k_precision.png}
  \small Precision@k

  \column{0.48\textwidth}
  \centering
  \includegraphics[width=\textwidth]{../images/fc_k_recall.png}
  \small Recall@k
\end{columns}

\end{frame}
}


% ------------------------------------------------------
% 4.5.2 BASEADO EM CONTEÚDO — ACURÁCIA
{
\bgbranco
\begin{frame}{Baseado em Conteúdo — Sensibilidade ao Top-\(k\) (Acurácia)}
\vspace{0.2cm}

\begin{columns}
  \column{0.48\textwidth}
  \centering
  \includegraphics[width=\textwidth]{../images/bc_rmse_k.png}
  \small RMSE

  \column{0.48\textwidth}
  \centering
  \includegraphics[width=\textwidth]{../images/bc_mae_k.png}
  \small MAE
\end{columns}

\end{frame}
}

% ------------------------------------------------------
% 4.5.2 BASEADO EM CONTEÚDO — RANKING
{
\bgbranco
\begin{frame}{Baseado em Conteúdo — Sensibilidade ao Top-\(k\) (Ranking)}
\vspace{0.2cm}

\begin{columns}
  \column{0.48\textwidth}
  \centering
  \includegraphics[width=\textwidth]{../images/bc_precision_k.png}
  \small Precision@k

  \column{0.48\textwidth}
  \centering
  \includegraphics[width=\textwidth]{../images/bc_recall_k.png}
  \small Recall@k
\end{columns}

\end{frame}
}


% ------------------------------------------------------
% 4.5.3 MODELO BASELINE — RESULTADOS
{
\bgbranco
\begin{frame}{Modelo Baseline — Resultados}
O modelo baseline é utilizado como referência mínima de desempenho,
permitindo avaliar os ganhos obtidos pelos modelos mais complexos.

\vspace{0.4cm}

\centering
\begin{table}
\small
\begin{tabular}{lcccc}
\toprule
Modelo & RMSE & MAE & Precision@k & Recall@k \\
\midrule
Baseline & 1,1141 & 0,8352 & 0,8489 & 0,1721 \\
\bottomrule
\end{tabular}
\end{table}

\vspace{0.3cm}

\begin{itemize}
  \item Não depende de vizinhança nem de parâmetros
  \item Serve como linha de base para comparação
\end{itemize}

\end{frame}
}


% =======================================================
% 4.6.1 COMPARAÇÃO ENTRE OS MODELOS
% =======================================================
{
\bgbranco
\begin{frame}{Desempenho dos Modelos Individuais}

\begin{table}[H]
    \centering
    \small
    \begin{tabular}{lcccc}
        \toprule
        \textbf{Modelo} & \textbf{RMSE} & \textbf{MAE} & \textbf{P@10} & \textbf{R@10} \\
        \midrule
        FC User--User (Pearson, $K=50$)       & 1,0895 & 0,8159 & 0,8709 & 0,1781 \\
        Baseline ($\mu + b_u + b_i$)  & 1,1141 & 0,8352 & 0,8489 & 0,1721 \\
        BC Item--Item (Jaccard, $K=30$)       & 1,1865 & 0,8877 & 0,7917 & 0,1595 \\
        \bottomrule
    \end{tabular}
\end{table}

\end{frame}
}

% ===============================================
% 4.6.2 INTERPRETAÇÃO DOS RESULTADOS
% ===============================================
{
\bgbranco
\begin{frame}{Limitações dos Modelos Individuais}

\begin{itemize}

    \item \textbf{Filtragem Colaborativa}
    \begin{itemize}
        \item Melhor desempenho preditivo
        \item Sensível à esparsidade e ao \textit{cold-start}
    \end{itemize}

    \item \textbf{Baseline}
    \begin{itemize}
        \item Estável e computacionalmente eficiente
        \item Baixa capacidade de personalização
    \end{itemize}

    \item \textbf{Baseado em Conteúdo}
    \begin{itemize}
        \item Maior cobertura do sistema
        \item Desempenho preditivo inferior
    \end{itemize}

\end{itemize}

\end{frame}
}

% =======================================================
% 4.6.3 MOTIVAÇÃO PARA O MODELO HÍBRIDO
% =======================================================
{
\bgbranco
\begin{frame}{Estratégia de Integração Híbrida}

\begin{itemize}

    \item Nenhuma abordagem isolada garante robustez em cenários reais.

    \item Proposta: combinar modelos para explorar vantagens complementares.

    \item Modelos implementados:
    \begin{itemize}
        \item \textbf{FC + BL}
        \item \textbf{FC + BC}
    \end{itemize}

    \item Objetivos:
    \begin{itemize}
        \item Melhorar desempenho preditivo
        \item Reduzir efeitos do \textit{cold-start}
        \item Aumentar estabilidade das recomendações
    \end{itemize}

\end{itemize}

\end{frame}
}


% ======================================================
% 4.7 MODELO HÍBRIDO — ACURÁCIA
% ======================================================
{
\bgbranco
\begin{frame}{Modelo Híbrido — Sensibilidade ao Parâmetro \(\beta\) (Acurácia)}
\vspace{0.2cm}

\begin{columns}
  \column{0.48\textwidth}
  \centering
  \includegraphics[width=\textwidth]{../images/hybrid_rmse_beta.png}
  \small RMSE

  \column{0.48\textwidth}
  \centering
  \includegraphics[width=\textwidth]{../images/hybrid_mae_beta.png}
  \small MAE
\end{columns}

\end{frame}
}

% ======================================================
% 4.7 MODELO HÍBRIDO — RANKING
% ======================================================
{
\bgbranco
\begin{frame}{Modelo Híbrido — Sensibilidade ao Parâmetro \(\beta\) (Ranking)}
\vspace{0.2cm}

\begin{columns}
  \column{0.48\textwidth}
  \centering
  \includegraphics[width=\textwidth]{../images/hybrid_precision_beta.png}
  \small Precision@k

  \column{0.48\textwidth}
  \centering
  \includegraphics[width=\textwidth]{../images/hybrid_recall_beta.png}
  \small Recall@k
\end{columns}

\end{frame}
}


% ======================================================
% 4.8 COMPARAÇÃO FINAL DOS MODELOS
% ======================================================
{
\bgbranco
\begin{frame}{Comparação Final — Acurácia}
A comparação entre os modelos evidencia diferenças no desempenho
preditivo segundo as métricas de acurácia.

\vspace{0.3cm}

\centering
\includegraphics[width=0.8\textwidth]{../images/comparacao_modelos_acuracia.png}
\end{frame}
}

{
\bgbranco
\begin{frame}{Comparação Final — Ranqueamento}
As métricas de ranqueamento permitem avaliar a qualidade das listas
de recomendação geradas pelos modelos.

\vspace{0.3cm}

\centering
\includegraphics[width=0.8\textwidth]{../images/comparacao_modelos_ranking.png}
\end{frame}
}

\section{Conclusão}
\sectionframe{Conclusão}

% ------------------------------------------------------
% CONCLUSÃO
% ------------------------------------------------------
{
\bgbranco
\begin{frame}{Conclusão}

A avaliação conjunta de erro e ranqueamento indica que o modelo híbrido FC+BC ($\beta = 0{,}7$) apresenta o melhor compromisso entre desempenho e aplicabilidade.

\vspace{1cm}

\begin{itemize}
    \item Erros preditivos (MAE e RMSE) competitivos,
    \item Redução do problema de \textit{cold-start} via componente Baseado em Conteúdo,
    \item Melhor qualidade do ranking (Precision@10 e Recall@10).
\end{itemize}

\end{frame}
}


% ------------------------------------------------------
% LIMITAÇÕES E TRABALHOS FUTUROS
% ------------------------------------------------------
{
\bgbranco
\begin{frame}{Limitações e Trabalhos Futuros}

\begin{itemize}
    \item Avaliação restrita a ambiente offline
    \item Baseline heurístico por limitações computacionais
    \item Extensão para Baseline regularizado ou Fatoração de Matrizes
    \item Enriquecimento do conteúdo dos itens (atributos adicionais)
\end{itemize}

\end{frame}
}


% ------------------------------------------------------
% SLIDE FINAL
% ------------------------------------------------------
{
\bgfundo
\begin{frame}[plain]
\vspace{1cm}
\centering
\Huge Obrigada!
\end{frame}
}


% ======================================================
% BIBTEX "SILENCIADO" (mantido por segurança)
% ======================================================
\nocite{*}
\bibliographystyle{plain}
\bibliography{dummy}

\end{document}
