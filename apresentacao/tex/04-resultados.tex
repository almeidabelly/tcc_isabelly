\section{Resultados}
\sectionframe{Resultados}

% ======================================================
% 4.1 IMPLEMENTAÇÃO E ANÁLISE PRÁTICA
% ======================================================
{
\bgbranco
\begin{frame}{Implementação e Análise Prática}
Este capítulo apresenta os resultados obtidos a partir da implementação
dos modelos de recomendação descritos na metodologia.

\vspace{0.3cm}

\begin{itemize}
  \item Análise exploratória e pré-processamento dos dados
  \item Definição da matriz de avaliações
  \item Treinamento, parametrização e avaliação dos modelos
\end{itemize}
\end{frame}
}

% ======================================================
% 4.2 ANÁLISE EXPLORATÓRIA E PRÉ-PROCESSAMENTO
% ======================================================
{
\bgbranco
\begin{frame}{Distribuição das Notas}
A distribuição das avaliações permite compreender o comportamento
geral das notas atribuídas pelos usuários.

%\vspace{0.3cm}

\centering
\includegraphics[width=0.65\textwidth]{../images/hist_notas.png}
\end{frame}
}

{
\bgbranco
\begin{frame}{Distribuição de Avaliações por Usuário}
A maioria dos usuários avalia apenas um número reduzido de jogos,
o que contribui para a esparsidade da base.

%\vspace{0.3cm}

\centering
\includegraphics[width=0.65\textwidth]{../images/dist_avaliacoes_usuario.png}
\end{frame}
}

{
\bgbranco
\begin{frame}{Distribuição de Avaliações por Jogo}
Observa-se concentração de avaliações em um subconjunto reduzido
de jogos, indicando assimetria de popularidade.

%\vspace{0.3cm}

\centering
\includegraphics[width=0.65\textwidth]{../images/dist_avaliacoes_jogo.png}
\end{frame}
}

% ======================================================
% 4.3 DEFINIÇÃO DA MATRIZ R
% ======================================================
{
\bgbranco
\begin{frame}{Matriz de Avaliações \(R\)}
Após o pré-processamento, os dados são organizados em uma matriz
usuário--jogo \(R\), utilizada como base para os modelos de recomendação.

\vspace{0.3cm}

\begin{itemize}
  \item Linhas representam usuários
  \item Colunas representam jogos
  \item Entradas correspondem às avaliações observadas
\end{itemize}
\end{frame}
}

% ======================================================
% 4.4 DIVISÃO DOS DADOS
% ======================================================
{
\bgbranco
\begin{frame}{Divisão dos Dados em Treino e Teste}
Os experimentos foram conduzidos por meio de uma divisão dos dados
em conjuntos de treinamento e teste.

\vspace{0.3cm}

\begin{itemize}
  \item Avaliação realizada em cenário offline
  \item Garantia de comparabilidade entre os modelos
\end{itemize}
\end{frame}
}

% ======================================================
% 4.5 TREINAMENTO DOS MODELOS
% ======================================================
% 4.5.1 FILTRAGEM COLABORATIVA — ACURÁCIA
{
\bgbranco
\begin{frame}{Filtragem Colaborativa — Sensibilidade ao Top-\(k\) (Acurácia)}
\vspace{0.2cm}

\begin{columns}
  \column{0.48\textwidth}
  \centering
  \includegraphics[width=\textwidth]{../images/fc_k_rmse.png}
  \small RMSE

  \column{0.48\textwidth}
  \centering
  \includegraphics[width=\textwidth]{../images/fc_k_mae.png}
  \small MAE
\end{columns}

\end{frame}
}

% 4.5.1 FILTRAGEM COLABORATIVA — RANKING
{
\bgbranco
\begin{frame}{Filtragem Colaborativa — Sensibilidade ao Top-\(k\) (Ranking)}
\vspace{0.2cm}

\begin{columns}
  \column{0.48\textwidth}
  \centering
  \includegraphics[width=\textwidth]{../images/fc_k_precision.png}
  \small Precision@k

  \column{0.48\textwidth}
  \centering
  \includegraphics[width=\textwidth]{../images/fc_k_recall.png}
  \small Recall@k
\end{columns}

\end{frame}
}


% ------------------------------------------------------
% 4.5.2 BASEADO EM CONTEÚDO — ACURÁCIA
{
\bgbranco
\begin{frame}{Baseado em Conteúdo — Sensibilidade ao Top-\(k\) (Acurácia)}
\vspace{0.2cm}

\begin{columns}
  \column{0.48\textwidth}
  \centering
  \includegraphics[width=\textwidth]{../images/bc_rmse_k.png}
  \small RMSE

  \column{0.48\textwidth}
  \centering
  \includegraphics[width=\textwidth]{../images/bc_mae_k.png}
  \small MAE
\end{columns}

\end{frame}
}

% ------------------------------------------------------
% 4.5.2 BASEADO EM CONTEÚDO — RANKING
{
\bgbranco
\begin{frame}{Baseado em Conteúdo — Sensibilidade ao Top-\(k\) (Ranking)}
\vspace{0.2cm}

\begin{columns}
  \column{0.48\textwidth}
  \centering
  \includegraphics[width=\textwidth]{../images/bc_precision_k.png}
  \small Precision@k

  \column{0.48\textwidth}
  \centering
  \includegraphics[width=\textwidth]{../images/bc_recall_k.png}
  \small Recall@k
\end{columns}

\end{frame}
}


% ------------------------------------------------------
% 4.5.3 MODELO BASELINE — RESULTADOS
{
\bgbranco
\begin{frame}{Modelo Baseline — Resultados}
O modelo baseline é utilizado como referência mínima de desempenho,
permitindo avaliar os ganhos obtidos pelos modelos mais complexos.

\vspace{0.4cm}

\centering
\begin{table}
\small
\begin{tabular}{lcccc}
\toprule
Modelo & RMSE & MAE & Precision@k & Recall@k \\
\midrule
Baseline & 1,1141 & 0,8352 & 0,8489 & 0,1721 \\
\bottomrule
\end{tabular}
\end{table}

\vspace{0.3cm}

\begin{itemize}
  \item Não depende de vizinhança nem de parâmetros
  \item Serve como linha de base para comparação
\end{itemize}

\end{frame}
}


% =======================================================
% 4.6.1 COMPARAÇÃO ENTRE OS MODELOS
% =======================================================
{
\bgbranco
\begin{frame}{Desempenho dos Modelos Individuais}

\begin{table}[H]
    \centering
    \small
    \begin{tabular}{lcccc}
        \toprule
        \textbf{Modelo} & \textbf{RMSE} & \textbf{MAE} & \textbf{P@10} & \textbf{R@10} \\
        \midrule
        FC User--User (Pearson, $K=50$)       & 1,0895 & 0,8159 & 0,8709 & 0,1781 \\
        Baseline ($\mu + b_u + b_i$)  & 1,1141 & 0,8352 & 0,8489 & 0,1721 \\
        BC Item--Item (Jaccard, $K=30$)       & 1,1865 & 0,8877 & 0,7917 & 0,1595 \\
        \bottomrule
    \end{tabular}
\end{table}

\end{frame}
}

% ===============================================
% 4.6.2 INTERPRETAÇÃO DOS RESULTADOS
% ===============================================
{
\bgbranco
\begin{frame}{Limitações dos Modelos Individuais}

\begin{itemize}

    \item \textbf{Filtragem Colaborativa}
    \begin{itemize}
        \item Melhor desempenho preditivo
        \item Sensível à esparsidade e ao \textit{cold-start}
    \end{itemize}

    \item \textbf{Baseline}
    \begin{itemize}
        \item Estável e computacionalmente eficiente
        \item Baixa capacidade de personalização
    \end{itemize}

    \item \textbf{Baseado em Conteúdo}
    \begin{itemize}
        \item Maior cobertura do sistema
        \item Desempenho preditivo inferior
    \end{itemize}

\end{itemize}

\end{frame}
}

% =======================================================
% 4.6.3 MOTIVAÇÃO PARA O MODELO HÍBRIDO
% =======================================================
{
\bgbranco
\begin{frame}{Estratégia de Integração Híbrida}

\begin{itemize}

    \item Nenhuma abordagem isolada garante robustez em cenários reais.

    \item Proposta: combinar modelos para explorar vantagens complementares.

    \item Modelos implementados:
    \begin{itemize}
        \item \textbf{FC + BL}
        \item \textbf{FC + BC}
    \end{itemize}

    \item Objetivos:
    \begin{itemize}
        \item Melhorar desempenho preditivo
        \item Reduzir efeitos do \textit{cold-start}
        \item Aumentar estabilidade das recomendações
    \end{itemize}

\end{itemize}

\end{frame}
}


% ======================================================
% 4.7 MODELO HÍBRIDO — ACURÁCIA
% ======================================================
{
\bgbranco
\begin{frame}{Modelo Híbrido — Sensibilidade ao Parâmetro \(\beta\) (Acurácia)}
\vspace{0.2cm}

\begin{columns}
  \column{0.48\textwidth}
  \centering
  \includegraphics[width=\textwidth]{../images/hybrid_rmse_beta.png}
  \small RMSE

  \column{0.48\textwidth}
  \centering
  \includegraphics[width=\textwidth]{../images/hybrid_mae_beta.png}
  \small MAE
\end{columns}

\end{frame}
}

% ======================================================
% 4.7 MODELO HÍBRIDO — RANKING
% ======================================================
{
\bgbranco
\begin{frame}{Modelo Híbrido — Sensibilidade ao Parâmetro \(\beta\) (Ranking)}
\vspace{0.2cm}

\begin{columns}
  \column{0.48\textwidth}
  \centering
  \includegraphics[width=\textwidth]{../images/hybrid_precision_beta.png}
  \small Precision@k

  \column{0.48\textwidth}
  \centering
  \includegraphics[width=\textwidth]{../images/hybrid_recall_beta.png}
  \small Recall@k
\end{columns}

\end{frame}
}


% ======================================================
% 4.8 COMPARAÇÃO FINAL DOS MODELOS
% ======================================================
{
\bgbranco
\begin{frame}{Comparação Final — Acurácia}
A comparação entre os modelos evidencia diferenças no desempenho
preditivo segundo as métricas de acurácia.

\vspace{0.3cm}

\centering
\includegraphics[width=0.8\textwidth]{../images/comparacao_modelos_acuracia.png}
\end{frame}
}

{
\bgbranco
\begin{frame}{Comparação Final — Ranqueamento}
As métricas de ranqueamento permitem avaliar a qualidade das listas
de recomendação geradas pelos modelos.

\vspace{0.3cm}

\centering
\includegraphics[width=0.8\textwidth]{../images/comparacao_modelos_ranking.png}
\end{frame}
}
