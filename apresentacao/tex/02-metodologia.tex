\section{Metodologia}
\sectionframe{Metodologia}

% ======================================================
% O QUE SÃO SISTEMAS DE RECOMENDAÇÃO
% ======================================================
{
\bgbranco
\begin{frame}{O que são Sistemas de Recomendação}
Sistemas de recomendação são ferramentas computacionais projetadas para
auxiliar usuários na descoberta de itens de interesse em ambientes com
grande volume de opções.

% \vspace{0.4cm}

% \begin{itemize}
%   \item Utilizados em domínios como filmes, músicas, produtos e jogos
%   \item Baseiam-se em dados históricos de interação dos usuários
%   \item Buscam reduzir a sobrecarga de escolha
% \end{itemize}

\end{frame}
}

% ======================================================
% OBJETIVOS DOS SISTEMAS DE RECOMENDAÇÃO
% ======================================================
{
\bgbranco
\begin{frame}{Objetivos dos Sistemas de Recomendação}
Os principais objetivos dos sistemas de recomendação incluem:

\begin{itemize}
  \item Sugerir itens relevantes de acordo com o perfil do usuário
  \item Melhorar a experiência e satisfação do usuário
  \item Aumentar o engajamento com a plataforma
  \item Auxiliar na descoberta de novos itens
\end{itemize}
\end{frame}
}

% ======================================================
% TIPOS E ABORDAGENS
% ======================================================
{
\bgbranco
\begin{frame}{Tipos e Abordagens de Sistemas de Recomendação}
Diferentes abordagens podem ser adotadas na construção de sistemas de
recomendação, cada uma com características específicas.

\vspace{0.4cm}

\begin{itemize}
  \item Filtragem Colaborativa
  \item Sistemas Baseados em Conteúdo
  \item Modelos Baseline
  \item Sistemas Híbridos
\end{itemize}
\end{frame}
}

% ======================================================
% FILTRAGEM COLABORATIVA
% ======================================================
{
\bgbranco
\begin{frame}{Filtragem Colaborativa}
A filtragem colaborativa explora padrões de avaliação entre usuários e itens,
assumindo que usuários com comportamentos semelhantes tendem a compartilhar
preferências.

\vspace{0.3cm}

O problema é modelado por uma matriz de avaliações
\( R \in \mathbb{R}^{|U| \times |I|} \), onde \( r_{ui} \) representa a
avaliação do usuário \( u \) para o item \( i \).

\vspace{0.3cm}

A matriz \(R\) é tipicamente esparsa.
\end{frame}
}

{
\bgbranco
\begin{frame}{Filtragem Colaborativa: Predição e Top-\(k\)}
A predição da avaliação é realizada a partir dos \(k\) vizinhos mais similares
(Top-\(k\)):

\[
\hat{r}_{ui} =
\bar{r}_u +
\frac{
\sum_{v \in N_k(u)} \text{sim}(u,v)\,(r_{vi} - \bar{r}_v)
}{
\sum_{v \in N_k(u)} |\text{sim}(u,v)|
}
\]

\vspace{0.3cm}

\textbf{Vantagens}
\begin{itemize}
  \item Boa capacidade de capturar padrões coletivos
\end{itemize}

\textbf{Desvantagens}
\begin{itemize}
  \item Sensível à esparsidade e ao problema de \textit{cold-start}
\end{itemize}
\end{frame}
}

% ======================================================
% SISTEMAS BASEADOS EM CONTEÚDO
% ======================================================
{
\bgbranco
\begin{frame}{Sistemas Baseados em Conteúdo}
Os sistemas baseados em conteúdo recomendam itens semelhantes àqueles
previamente avaliados pelo usuário, utilizando características descritivas
dos itens.

\vspace{0.3cm}

A predição é obtida considerando os \(k\) itens mais similares:

\[
\hat{r}_{ui} =
\frac{
\sum_{j \in N_k(i)} \text{sim}(i,j)\, r_{uj}
}{
\sum_{j \in N_k(i)} |\text{sim}(i,j)|
}
\]

\vspace{0.3cm}

\textbf{Vantagens}
\begin{itemize}
  \item Menor impacto do \textit{cold-start} de itens
\end{itemize}

\textbf{Desvantagens}
\begin{itemize}
  \item Dependência da qualidade dos atributos dos itens
\end{itemize}
\end{frame}
}

% ======================================================
% MODELO BASELINE
% ======================================================
{
\bgbranco
\begin{frame}{Modelo Baseline}
O modelo baseline fornece uma estimativa simples das avaliações,
considerando efeitos globais e individuais:

\[
\hat{r}_{ui} = \mu + b_u + b_i
\]

\vspace{0.3cm}

\textbf{Objetivo}
\begin{itemize}
  \item Servir como referência mínima de desempenho
\end{itemize}

\textbf{Vantagens}
\begin{itemize}
  \item Simples e computacionalmente eficiente
\end{itemize}

\textbf{Desvantagens}
\begin{itemize}
  \item Não captura interações complexas entre usuários e itens
\end{itemize}
\end{frame}
}

% ======================================================
% MODELO HÍBRIDO
% ======================================================
{
\bgbranco
\begin{frame}{Modelo Híbrido}
Os sistemas híbridos combinam diferentes abordagens de recomendação,
buscando explorar suas vantagens complementares.

\[
\hat{r}_{u,i}^{(H)} =
\beta\,\hat{r}_{u,i}^{(SR_1)} +
(1-\beta)\,\hat{r}_{u,i}^{(SR_2)}, 
\text{     onde $\beta \in [0,1]$}
\]
\vspace{0.3cm}

\begin{itemize}
  \item Integra duas abordagens distintas
  \item Reduz limitações individuais dos modelos
\end{itemize}
\end{frame}
}

% ======================================================
% MEDIDAS DE SIMILARIDADE
% ======================================================
{
\bgbranco
\begin{frame}{Medidas de Similaridade}
As medidas de similaridade variam conforme o tipo de sistema de recomendação.

\vspace{0.3cm}

\begin{itemize}
  \item \textbf{Correlação de Pearson}: adequada para filtragem colaborativa
  \item \textbf{Similaridade do Cosseno}: utilizada em vetores de atributos
  \item \textbf{Distância Euclidiana}: sensível à magnitude das avaliações
  \item \textbf{Coeficiente de Jaccard}: apropriado para dados binários
\end{itemize}
\end{frame}
}

% ======================================================
% MÉTRICAS DE ACURÁCIA
% ======================================================
{
\bgbranco
\begin{frame}{Métricas de Avaliação — Acurácia}

Avaliam o erro entre os valores reais observados e as previsões geradas pelos modelos.

\vspace{0.3cm}

\textbf{Erro Médio Absoluto (MAE)}
\[
\text{MAE} =
\frac{1}{N}\sum |r_{ui} - \hat{r}_{ui}|,
\]

{\footnotesize
penaliza todos os desvios de forma linear mantendo a unidade original da variável analisada.
}

\vspace{0.3cm}

\textbf{Raiz do Erro Quadrático Médio (RMSE)}
\[
\text{RMSE} =
\sqrt{\frac{1}{N}\sum (r_{ui} - \hat{r}_{ui})^2},
\]

{\footnotesize
Atribui maior peso a erros de grande magnitude, sendo sensível a previsões muito distantes dos valores reais.
}

\end{frame}
}


% ======================================================
% MÉTRICAS DE RANQUEAMENTO
% ======================================================
{
\bgbranco
\begin{frame}{Métricas de Avaliação — Ranqueamento}

Avaliam a capacidade do modelo em priorizar corretamente os itens mais relevantes entre as primeiras posições da recomendação.

\vspace{0.4cm}

\textbf{Precision@k}
\[
\text{Precision@k} =
\frac{| \text{itens relevantes} \cap \text{top-}k |}{k},\text{}
\]

{\footnotesize
Indica a proporção de itens relevantes entre os $k$ primeiros recomendados, refletindo a qualidade do topo da lista.
}

\vspace{0.35cm}

\textbf{Recall@k}
\[
\text{Recall@k} =
\frac{| \text{itens relevantes} \cap \text{top-}k |}{|\text{itens relevantes}|}
\]

{\footnotesize
Mede a capacidade do modelo em recuperar os itens relevantes disponíveis, considerando o conjunto total de itens de interesse do usuário.
}

\end{frame}
}

