%% Abstract.tex
% ---
% Abstract
% ---
\autor{Silva, M. J.}
\begin{resumo}[Abstract]
 \begin{otherlanguage*}{english}
	\begin{flushleft} 
		\setlength{\absparsep}{0pt} % ajusta o espaçamento dos parágrafos do resumo		
 		\SingleSpacing 
 		\imprimirautorabr~ ~\textbf{\imprimirtitleabstract}.	\imprimirdata.  \pageref{LastPage}p. 
		%Substitua p. por f. quando utilizar oneside em \documentclass
		%\pageref{LastPage}f.
		\imprimirtipotrabalho~-~\imprimirinstituicao, \imprimirlocal, 	\imprimirdata. 
 	\end{flushleft}
	\OnehalfSpacing 
   This work analyzes the application of Recommendation Systems in the context of modern board games using real-world data from the \textit{BoardGameGeek} platform. Different approaches are evaluated, including Collaborative Filtering, Content-Based models, and hybrid models, with the aim of comparing their predictive performance and ranking quality. The methodology comprises exploratory data analysis, construction of the user--item matrix, hyperparameter tuning, and evaluation using accuracy metrics (RMSE and MAE) and ranking metrics (Precision@k and Recall@k). The results indicate that hybrid models outperform individual approaches, with the combination of Collaborative Filtering and Content-Based methods providing a better balance between accuracy, recommendation quality, and mitigation of the \textit{cold-start} problem. It is concluded that integrating multiple sources of information is an effective strategy for Recommendation Systems applied to modern board games.

   \vspace{\onelineskip}
 
   \noindent 
   \textbf{Keywords}: Recommendation Systems. Board Games. Collaborative Filtering. Baseline Method. Content-Based Filtering. Hybrid Models.
 \end{otherlanguage*}
\end{resumo}
